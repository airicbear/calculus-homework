\documentclass[12pt]{article}
\usepackage[margin=1in]{geometry} 
\usepackage{amsmath,amsthm,amssymb,amsfonts,mathtools}

\newenvironment{problem}[2][]{
    \begin{trivlist}
        \item[
            {\bfseries #1}
            {\bfseries #2.}
        ]
}{\end{trivlist}}

\pagenumbering{gobble}

% --------------------------------------------------
% CUSTOM COMMANDS
% -------------------------------------------------- 

\newcommand{\assignment}{Pg. 463 \#17, 26-39, 41-43, 45-53 odd}
\newcommand{\name}{Eric Nguyen}
\newcommand{\duedate}{2019-03-27}
\newcommand{\details}{\textbf{\\\name \\\duedate \\\assignment}}

\newcommand{\descprob}[1]{\hfill\break #1}
\newcommand{\plugin}[2]{\left(\left({#1}\right) - \left({#2}\right)\right)}
\newcommand{\setuv}[4]{
\left[
\begin{alignedat}{2}
u &= #1 &\quad v &= #3 \\
du &= #2 &\quad dv &= #4 \\
\end{alignedat}
\right]  = uv - \int v ~ du \\
&= \intbp{#1}{#3}{#2}
}
\newcommand{\subu}[2]{
\left[
\begin{alignedat}{1}
u &= #1 \\
du &= #2 \\
\end{alignedat}
\right] 
}
\newcommand{\intbp}[3]{\left(#1\right) \left(#2\right) - \int \left(#2\right) \left(#3\right)}
\newcommand{\descmeet}{\shortintertext{\quad Find where the graphs meet:}}
\newcommand{\deschigh}{\shortintertext{\quad Find the graph that is higher between the interval:}}
\newcommand{\descarea}{\shortintertext{\quad Find the area of the difference of the two graphs between the interval:}}

% --------------------------------------------------  
% TABULAR INTEGRATION
% -------------------------------------------------- 

\usepackage{booktabs}
\usepackage{xparse}
\usepackage{tikz}
\usetikzlibrary{calc}

\tikzset{Arrow Style/.style={text=black, font=\boldmath}}

\newcommand{\tikzmark}[1]{%
    \tikz[overlay, remember picture, baseline] \node (#1) {};%
}

\newcommand*{\XShift}{0.5em}
\newcommand*{\YShift}{0.5ex}

\NewDocumentCommand{\DrawArrow}{s O{} m m m}{%
    \begin{tikzpicture}[overlay,remember picture]
        \draw[->, thick, Arrow Style, #2] 
                ($(#3.west)+(\XShift,\YShift)$) -- 
                ($(#4.east)+(-\XShift,\YShift)$)
        node [midway,above] {#5};
    \end{tikzpicture}%
}

\begin{document}

\details

\begin{problem}{17}
Give an interpretation of the shaded region.
\begin{align}
\text{Total words in} ~ t ~ \text{minutes}.
\end{align}
\end{problem}

\begin{problem}{26}
Find the area of the region bounded by $y = 3x^2$ and $y = 9x$
\begin{align}
\descmeet
3x^2 &= 9x \\
3x^2 - 9x &= 0 \\
x^2 - 3x &= 0 \\
x(x - 3) &= 0 \\
x = 0 &\And x = 3 \\
\deschigh
3(1)^2 &< 9(1) \\
\descarea
\int_0^3 9x ~ dx - \int_0^3 3x^2 ~ dx &= \frac{9}{2} x^2 ~ \bigg|_0^3 - x^3 ~ \bigg|_0^3 \\
&= \frac{9}{2} (3)^2 - (3)^3 \\
&= \frac{81}{2} - \frac{54}{2} \\
&= 13 \frac{1}{2}
\end{align}
\end{problem}

\descprob{Evaluate using substitution.}

\begin{problem}{27}
$\displaystyle\int x^3 e^{x^4} ~ dx$
\begin{align}
&= \subu{x^4}{4x^3 ~ dx} = \frac{1}{4} \int e^u ~ du \\
&= \frac{1}{4} e^{x^4} + C
\end{align}
\end{problem}

\begin{problem}{28}
$\displaystyle\int \frac{24t^5}{4t^6 + 3} ~ dt$
\begin{align}
&= \subu{4t^6}{24t^5 ~ dt} = \int \frac{du}{u + 3} \\
&= \ln \left(4t^6 + 3\right) + C
\end{align}
\end{problem}

\begin{problem}{29}
    $\displaystyle\int \frac{\ln \left(4x\right)}{2x} ~ dx$
\begin{align}
&= \subu{\ln \left(4x\right)}{\frac{1}{x} ~ dx} = 2 \int u ~ du \\
&= 2 \ln \left(4x\right) + C
\end{align}
\end{problem}

\begin{problem}{30}
$\displaystyle\int 2e^{-3x} ~ dx$
\begin{align}
&= - \frac{2}{3} e^{-3x} + C
\end{align}
\end{problem}

\descprob{Evaluate using integration by parts.}

\begin{problem}{31}
$\displaystyle\int 3x e^{3x} ~ dx$
\begin{align}
&= 3 \int x e^{3x} ~ dx \\
\int x e^{3x} ~ dx &= \setuv{x}{dx}{\frac{e^{3x}}{3}}{e^{3x} ~ dx} \\
\int 3x e^{3x} ~ dx &= 3 \left(\frac{1}{3} xe^{3x} - \frac{1}{9} e^{3x}\right) \\
&= xe^{3x} - \frac{1}{3} e^{3x} + C
\end{align}
\end{problem}

\begin{problem}{32}
$\displaystyle\int \ln \sqrt[3]{x^2} ~ dx$
\begin{align}
&= \frac{2}{3} \int \ln x ~ dx \\
\int \ln x ~ dx &= \setuv{\ln x}{\frac{1}{x} ~ dx}{x}{dx} \\
&= x \ln \left(x\right) - x \\
\int \ln \sqrt[3]{x^2} ~ dx &= \frac{2}{3} \left(x \ln \left(x\right) - x\right) + C
\end{align}
\end{problem}

\begin{problem}{33}
$\displaystyle\int 3x^2 \ln x ~ dx$
\begin{align}
&= 3 \int x^2 \ln x ~ dx \\
\int x^{2} \ln x ~ dx &= \setuv{\ln x}{\frac{1}{x} ~ dx}{\frac{x^3}{3}}{x^2 ~ dx} \\
&= \frac{1}{3} x^3 \ln \left(x\right) - \frac{x^3}{9} \\
\int 3x^2 \ln x ~ dx &= x^3 \ln \left(x\right) - \frac{x^3}{3} + C
\end{align}
\end{problem}

\descprob{Evaluate using tables of integration.}

\begin{problem}{34}
$\displaystyle\int \frac{1}{49 - x^2} ~ dx$
\begin{align}
&= \frac{1}{14} \ln \left|\frac{7 + x}{7 - x}\right| + C
\end{align}
\end{problem}

\begin{problem}{35}
$\displaystyle\int x^2 e^{5x} ~ dx$
\begin{align}
&= \frac{x^2 e^{5x}}{5} - \frac{2}{5} \int x e^{5x} ~ dx \\
\int x e^{5x} ~ dx &= \frac{1}{25} \cdot e^{5x} \left(5x - 1\right) \\
&= \frac{1}{5} x e^{5x} - \frac{1}{25} e^{5x} \\
\int x^2 e^{5x} ~ dx &= \frac{1}{5} x^2 e^{5x} - \frac{2}{5} \left(\frac{1}{5} x e^{5x} - \frac{1}{25} e^{5x}\right) \\
&= \frac{1}{5} x^2 e^{5x} - \frac{2}{25} e^{5x} + \frac{2}{125} e^{5x} + C
\end{align}
\end{problem}

\begin{problem}{36}
$\displaystyle\int \frac{x}{7x + 1} ~ dx$
\begin{align}
&= \frac{1}{49} + \frac{x}{7} - \frac{1}{49} \ln \left|1 + 7x\right| + C
\end{align}
\end{problem}

\begin{problem}{37}
$\displaystyle\int \frac{dx}{\sqrt{x^2 - 36}}$
\begin{align}
&= \ln \left|x + \sqrt{x^2 + 6}\right| + C
\end{align}
\end{problem}

\begin{problem}{38}
$\displaystyle\int x^6 \ln x ~ dx$
\begin{align}
&= x^7 \left[\frac{\ln x}{7} - \frac{1}{49}\right] + C
\end{align}
\end{problem}

\begin{problem}{39}
$\displaystyle\int xe^{8x} ~ dx$
\begin{align}
&= \frac{1}{64} e^{8x} \left(8x - 1\right) + C
\end{align}
\end{problem}

\descprob{Word problems.}

\begin{problem}{41}
Find the average value of $y = xe^{-x}$ over [0, 2]
\begin{align}
y_{\text{av}} &= \frac{1}{2} \int_0^2 xe^{-x} ~ dx \\
\int_0^2 xe^{-x} ~ dx &= 
\renewcommand{\arraystretch}{1.5}
\begin{array}{c @{\hspace*{1.0cm}} c}
D & I \\
\cmidrule{1-2}
x \tikzmark{D1} & \tikzmark{I1}e^{-x} \\
1 \tikzmark{D2} & \tikzmark{I2}-e^{-x} \\
0 \tikzmark{D3} & \tikzmark{I3}e^{-x}
\end{array} = - x e^{-x} - e^{-x} \\
y_{av} &= \frac{1}{2} \left(- x e^{-x} - e^{-x} ~ \bigg|_0^2\right) \\
&= \frac{1}{2} \plugin{-2e^{-2} - e^{-2}}{-1} \\
&= \frac{1}{2} \left(1 - 3e^{-2}\right) \approx 0.297
\end{align}
\DrawArrow{D1}{I2}{$+$}
\DrawArrow{D2}{I3}{$-$}
\end{problem}

\begin{problem}{42}
$\displaystyle\int_0^4 3t^2 + 2t ~ dt$
\begin{align}
&= 3 \int_0^4 t^2 ~ dt + 2 \int_0^4 t ~ dt \\
&= t^3 + t^2 ~ \bigg|_0^4 \\
&= 80 ~ \text{mi}
\end{align}
\end{problem}

\begin{problem}{43}
$\displaystyle\int_0^4 3e^{3t} ~ dt$
\begin{align}
&= e^{3t} ~ \bigg|_0^4 \\
&= e^{12} - 1 \approx \$162,753.79
\end{align}
\end{problem}

\descprob{Integrate using any method.}

\begin{problem}{45}
$\displaystyle\int \frac{12t^2}{4t^3 + 7} ~ dt$
\begin{align}
&= \subu{4t^3}{12t^3 ~ dt} = \int \frac{du}{u + 7} \\
&= \ln \left|4t^3 + 7\right| + C \\
\end{align}
\end{problem}

\begin{problem}{47}
$\displaystyle\int 5x^4 e^{x^5} ~ dx$
\begin{align}
&= \subu{x^5}{5x^4 ~ dx} = \int e^u ~ du \\
&= e^{x^5} + C
\end{align}
\end{problem}

\begin{problem}{49}
$\displaystyle\int t^7 \left(t^8 + 3\right)^{11} ~ dt$
\begin{align}
&= \subu{t^8 + 3}{8t^7 ~ dt} = \frac{1}{7} \int u^{11} ~ du \\
&= \frac{\left(t^8 + 3\right)^{12}}{84} + C
\end{align}
\end{problem}

\begin{problem}{51}
$\displaystyle\int x \ln \left(8x\right) ~ dx$
\begin{align}
&= - \frac{1}{4} x^2 + \frac{1}{2} x^2 \ln \left(8x\right) + C
\end{align}
\end{problem}

\begin{problem}{53}
$\displaystyle\int \frac{dx}{e^x + 2}$
\begin{align}
&= \int \frac{1 + e^x}{e^x + 2} - \frac{e^x}{e^x + 2} ~ dx \\
&= \int \frac{1 + e^x}{e^x + 2} ~ dx - \int \frac{e^x}{e^x + 2} ~ dx \\
\int \frac{1 + e^x}{e^x + 2} ~ dx &= \int \frac{e^x}{e^x + 2} ~ dx + \int \frac{1}{e^x + 2} ~ dx \\
\int \frac{e^x}{e^x + 2} ~ dx &= \subu{e^x + 2}{e^x} = \frac{du}{u} \\
&= \ln \left(e^x + 2\right) \\
\int \frac{dx}{e^x + 2} &= \int \frac{dx}{e^x + 2}
\end{align}
\end{problem}

\end{document}
