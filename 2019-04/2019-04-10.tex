\documentclass[12pt]{article}
\usepackage[margin=1in]{geometry} 
\usepackage{amsmath,amsthm,amssymb,amsfonts,mathtools}
\usepackage[makeroom]{cancel}

\newenvironment{problem}[2][]{
    \begin{trivlist}
        \item[
            {\bfseries #1}
            {\bfseries #2.}
        ]
}{\end{trivlist}}

\pagenumbering{gobble}

% --------------------------------------------------
% CUSTOM COMMANDS
% -------------------------------------------------- 

\newcommand{\assignment}{Pg. 524 \#1-31 eoo, 42-43, 45}
\newcommand{\name}{Eric Nguyen}
\newcommand{\duedate}{2019-04-10}
\newcommand{\details}{\noindent\textbf{\name \\\duedate \\\assignment}}

\newcommand{\descprob}[1]{\hfill\break #1}
\newcommand{\plugin}[2]{\left(\left({#1}\right) - \left({#2}\right)\right)}
\newcommand{\setuv}[4]{
\left[
\begin{alignedat}{2}
u &= #1 &\quad v &= #3 \\
du &= #2 &\quad dv &= #4 \\
\end{alignedat}
\right]  = uv - \int v ~ du \\
&= \intbp{#1}{#3}{#2}
}
\newcommand{\subu}[2]{
\left[
\begin{alignedat}{1}
u &= #1 \\
du &= #2 \\
\end{alignedat}
\right] 
}
\newcommand{\intbp}[3]{\left(#1\right) \left(#2\right) - \int \left(#2\right) \left(#3\right)}
\newcommand{\descmeet}{\shortintertext{\quad Find where the graphs meet:}}
\newcommand{\deschigh}{\shortintertext{\quad Find the graph that is higher between the interval:}}
\newcommand{\descarea}{\shortintertext{\quad Find the area of the difference of the two graphs between the interval:}}

% --------------------------------------------------  
% TABULAR INTEGRATION
% -------------------------------------------------- 

\usepackage{booktabs}
\usepackage{xparse}
\usepackage{tikz}
\usetikzlibrary{calc}

\tikzset{Arrow Style/.style={text=black, font=\boldmath}}

\newcommand{\tikzmark}[1]{%
    \tikz[overlay, remember picture, baseline] \node (#1) {};%
}

\newcommand*{\XShift}{0.5em}
\newcommand*{\YShift}{0.5ex}

\NewDocumentCommand{\DrawArrow}{s O{} m m m}{%
    \begin{tikzpicture}[overlay,remember picture]
        \draw[->, thick, Arrow Style, #2] 
                ($(#3.west)+(\XShift,\YShift)$) -- 
                ($(#4.east)+(-\XShift,\YShift)$)
        node [midway,above] {#5};
    \end{tikzpicture}%
}

\begin{document}

\details

\begin{problem}{1}
$y' = 5x^4$
\begin{align}
&= x^5 + C \\
&= x^5,
\quad x^5 - 1,
\quad x^5 + 1
\end{align}
\end{problem}

\begin{problem}{5}
$y' = \frac{8}{x} - x^2 + x^5$
\begin{align}
&= 8 \ln |x| - \frac{x^3}{3} + \frac{x^6}{6} + C \\
&= 8 \ln |x| - \frac{x^3}{3} + \frac{x^6}{6},
\quad 8 \ln |x| - \frac{x^3}{3} + \frac{x^6}{6} - 1,
\quad 8 \ln |x| - \frac{x^3}{3} + \frac{x^6}{6} + 1
\end{align}
\end{problem}

\begin{problem}{9}
$f'(x) = x^{2/3} - x; \quad f(1) = -6$
\begin{align}
&= \frac{3}{5} x^{5/3} - \frac{x^2}{2} + C \\
-6 &= \frac{3}{5} \left(1\right)^{5/3} - \frac{(1)^2}{2} + C \\
C &= -6 - \frac{3}{5} + \frac{1}{2} = -\frac{61}{10} \\
f(x) &= \frac{3}{5} x^{5/3} - \frac{x^2}{2} - \frac{61}{10}
\end{align}
\end{problem}

\begin{problem}{13}
Show that $y = e^x + 3x e^x$ is a solution of $y'' - 2y' + y = 0$.
\begin{align}
y' &= e^x + 3\left(xe^x - e^x\right) \\
y'' &= e^x + 3\left(xe^x - e^x - e^x\right) \\
\left(e^x + 3\left(xe^x - e^x - e^x\right)\right) - 2\left(e^x + 3\left(xe^x - e^x\right)\right) + \left(e^x + 3x e^x\right) &= 0 \\
\cancel{e^x} + \cancel{3xe^x} - \cancel{3e^x - 3e^x} - \cancel{2e^x} - \cancel{6xe^x} + \cancel{6e^x} + \cancel{e^x} + \cancel{3xe^x} &= 0 \quad \checkmark
\end{align}
\end{problem}

\begin{problem}{17}
$3y^2 \frac{dy}{dx} = 8x$
\begin{align}
3 \int y^2 ~ dy &= 8 \int x ~ dx \\
y^3 &= 4x^2 + C \\
y &= \sqrt[3]{4x^2 + C}
\end{align}
\end{problem}

\begin{problem}{21}
$\frac{dy}{dx} = \frac{6}{y}$
\begin{align}
\int y ~ dy &= 6 \int dx \\
\frac{y^2}{2} &= 6x + C \\
y &= \sqrt{12x + C}
\end{align}
\end{problem}

\begin{problem}{25}
$y' = 5y^{-2}; \quad y = 3$ when $x = 2$
\begin{align}
\frac{dy}{dx} &= 5y^{-2} \\
\frac{1}{5} \int y^2 ~ dy &= \int dx \\
\frac{1}{15} y^3 &= x + C \\
y &= \sqrt[3]{15x + C} \\
3 &= \sqrt[3]{15(2) + C} \\
27 &= 30 + C \\
C &= -3 \\
y &= \sqrt[3]{15x - 3}
\end{align}
\end{problem}

\begin{problem}{29}
$\frac{dP}{dt} = 2P$
\begin{align}
\frac{1}{2} \int \frac{dP}{P} &= \int dt \\
\frac{1}{2} \ln \left|P\right| &= t + C \\
P &= C_1e^{2t}, \quad \text{where } C_1 = \pm e^C
\end{align}
\end{problem}

\begin{problem}{42}
$\frac{dP}{dt} = kP$
\begin{align}
\shortintertext{(a)}
\frac{1}{k} \int \frac{dP}{P} &= \int dt \\
\frac{1}{k} \ln \left|P\right| &= t + C \\
P &= C_1e^{kt}
\shortintertext{(b)}
P_0 &= 0 \\
P &= P_0 e^{kt}
\end{align}
\end{problem}

\begin{problem}{43}
$\frac{dR}{dS} = k \cdot \frac{R}{S}$
\begin{align}
S \cdot dR &= k \cdot R \cdot dS \\
\int \frac{dR}{R} &= k \int \frac{dS}{S} \\
\ln \left|R\right| &= k \ln \left|S\right| + C \\
R &= C_1 e^{k \ln S} \\
&= C_1 S^k, \quad \text{where } C_1 = e^C
\end{align}
\end{problem}

\begin{problem}{45}
$e^{-1/x} \cdot \frac{dy}{dx} = x^{-2} \cdot y^2$
\begin{align}
\int y^{-2} ~ dy &= \int x^{-2} e^{x^{-1}} ~ dx \\
\int x^{-2} e^{x^{-1}} ~ dx &= \subu{x^{-1}}{-x^{-2} ~ dx} = - \int e^u ~ du \\
-y^{-1} &= -e^{x^{-1}} + C \\
y &= \frac{1}{e^{1/x} - C}
\end{align}
\end{problem}

\end{document}
