\documentclass[12pt]{article}
\usepackage[margin=1in]{geometry} 
\usepackage{amsmath,amsthm,amssymb,amsfonts,mathtools,tikz}
\usepackage[makeroom]{cancel}

\newenvironment{problem}[2][]{
    \begin{trivlist}
        \item[
            {\bfseries #1}
            {\bfseries #2.}
        ]
}{\end{trivlist}}

\pagenumbering{gobble}

% --------------------------------------------------
% CUSTOM COMMANDS
% -------------------------------------------------- 

\newcommand{\assignment}{Pg. 545 \#1-23 odd}
\newcommand{\name}{Eric Nguyen}
\newcommand{\duedate}{2019-04-12}
\newcommand{\details}{\noindent\textbf{\name \\\duedate \\\assignment}}

\newcommand{\descprob}[1]{\hfill\break #1}
\newcommand{\plugin}[2]{\left(\left({#1}\right) - \left({#2}\right)\right)}
\newcommand{\setuv}[4]{
\left[
\begin{alignedat}{2}
u &= #1 &\quad v &= #3 \\
du &= #2 &\quad dv &= #4 \\
\end{alignedat}
\right]  = uv - \int v ~ du \\
&= \intbp{#1}{#3}{#2}
}
\newcommand{\subu}[2]{
\left[
\begin{alignedat}{1}
u &= #1 \\
du &= #2 \\
\end{alignedat}
\right] 
}
\newcommand{\intbp}[3]{\left(#1\right) \left(#2\right) - \int \left(#2\right) \left(#3\right)}
\newcommand{\descmeet}{\shortintertext{\quad Find where the graphs meet:}}
\newcommand{\deschigh}{\shortintertext{\quad Find the graph that is higher between the interval:}}
\newcommand{\descarea}{\shortintertext{\quad Find the area of the difference of the two graphs between the interval:}}

% --------------------------------------------------  
% TABULAR INTEGRATION
% -------------------------------------------------- 	

\usepackage{booktabs}
\usepackage{xparse}
\usepackage{tikz}
\usetikzlibrary{calc}

\tikzset{Arrow Style/.style={text=black, font=\boldmath}}

\newcommand{\tikzmark}[1]{%
    \tikz[overlay, remember picture, baseline] \node (#1) {};%
}

\newcommand*{\XShift}{0.5em}
\newcommand*{\YShift}{0.5ex}

\NewDocumentCommand{\DrawArrow}{s O{} m m m}{
    \begin{tikzpicture}[overlay,remember picture]
        \draw[->, thick, Arrow Style, #2] 
                ($(#3.west)+(\XShift,\YShift)$) -- 
                ($(#4.east)+(-\XShift,\YShift)$)
        node [midway,above] {#5};
    \end{tikzpicture}
}

\begin{document}

\details

\begin{problem}{1}
For $f\left(x,y\right) = x^2 - 3xy$, find $f\left(0, -2\right), f\left(2, 3\right)$, and $f\left(10, -5\right)$.
\begin{align}
f\left(0, -2\right) &= \left(0\right)^2 - 3\left(0\right)\left(-2\right) = 0 \\
f\left(2, 3\right) &= \left(2\right)^2 - 3\left(2\right)\left(3\right) = -14 \\
f\left(10, -5\right) &= \left(10\right)^2 - 3\left(10\right)\left(-5\right) = 250
\end{align}
\end{problem}

\begin{problem}{3}
For $f\left(x,y\right) = 3^x + 7xy$, find $f\left(0, -2\right), f\left(-2, 1\right)$, and $f\left(2, 1\right)$.
\begin{align}
f\left(0, -2\right) &= 3^{\left(0\right)} + 7\left(0\right)\left(-2\right) = 1 \\
f\left(-2, 1\right) &= 3^{\left(-2\right)} + 7\left(-2\right)\left(1\right) = -13 \frac{8}{9}  \\
f\left(2, 1\right) &= 3^{\left(2\right)} + 7\left(2\right)\left(1\right) = 23
\end{align}
\end{problem}

\begin{problem}{5}
For $f\left(x,y\right) = \ln x + y^3$, find $f\left(e, 2\right), f\left(e^2, 4\right)$, and $f\left(e^3, 5\right)$.
\begin{align}
f\left(e, 2\right) &= \ln \left(e\right) + (2)^3 = 9 \\
f\left(e^2, 4\right) &= 2 \ln \left(e^{\cancel{2}}\right) + (4)^3 = 66 \\
f\left(e^3, 5\right) &= 3 \ln \left(e^{\cancel{3}}\right) + (5)^3 = 128
\end{align}
\end{problem}

\begin{problem}{7}
For $f\left(x,y,z\right) = x^2 - y^2 + z^2$, find $f\left(-1, 2, 3\right)$ and $f\left(2, -1, 3\right)$.
\begin{align}
f\left(-1, 2, 3\right) &= \left(-1\right)^2 - \left(2\right)^2 + \left(3\right)^2 = 6 \\
f\left(2, -1, 3\right) &= \left(2\right)^2 - \left(-1\right)^2 + \left(3\right)^2 = 12
\end{align}
\end{problem}

\begin{problem}{9}
$R\left(P, E\right) = \frac{P}{E}$
\begin{align}
R\left(32.03, 1.25\right) &= \frac{32.03}{1.25} \approx 25.62
\end{align}
\end{problem}

\begin{problem}{11}
$C_2 = \left(\frac{V_2}{V_1}\right)^{0.6} C_1$
\begin{align}
C_2 = \left(\frac{160,000}{80,000}\right)^{0.6} \left(100,000\right) \approx \$151,571.66
\end{align}
\end{problem}

\begin{problem}{13}
$S\left(a,d,V\right) = \frac{aV}{0.51d^2}$
\begin{align}
S\left(0.78,100, 1.6 \times 10^6\right) = \frac{\left(0.78\right)\left(1.6 \times 10^6\right)}{0.51 \left(100\right)^2} &\approx 244.7 \,\text{mph}
\end{align}
\end{problem}

\begin{problem}{15}
$S\left(h,w\right) = 0.024265h^{0.3964} w^{0.5378}$
\begin{align}
S\left(150,80\right) = 0.024265\left(150\right)^{0.3964} \left(80\right)^{0.5378} \approx 1.87 \,\text{m}^2
\end{align}
\end{problem}

\begin{problem}{17}
For the tornado described in Exercise 13, if the wind speed measures 200 mph, how far from the center was the measurement taken?
\begin{align}
200 &= \frac{\left(0.78\right)\left(1.6 \times 10^6\right)}{0.51d^2} \\
d &= \sqrt{\frac{\left(0.78\right)\left(1.6 \times 10^6\right)}{0.51 \cdot 200}} \\
&\approx 110.6 \,\text{ft}
\end{align}
\end{problem}

\begin{problem}{19}
Explain the difference between a function of two variables and a function of one variable.\bigskip\\

\hangindent=\parindent
A function of two variables has two inputs, such as $x$ and $y$, while a function of one variable only has one input.
\end{problem}

\bigskip 

\noindent For 21 and 23: $W\left(v, T\right) = 91.4 - \frac{\left(10.45 + 6.68 \sqrt{v} - 0.447v\right)\left(457 - 5T\right)}{110}$

\begin{problem}{21}
$W(25, 30)$
\begin{align}
= 91.4 - \frac{\left(10.45 + 6.68 \sqrt{25} - 0.447\left(25\right)\right)\left(457 - 5\left(30\right)\right)}{110} \approx 0^\circ \text{F}
\end{align}
\end{problem}

\begin{problem}{23}
$W(40, 20)$
\begin{align}
= 91.4 - \frac{\left(10.45 + 6.68 \sqrt{40} - 0.447\left(40\right)\right)\left(457 - 5\left(20\right)\right)}{110} \approx -22^\circ \text{F}
\end{align}
\end{problem}

\end{document}
