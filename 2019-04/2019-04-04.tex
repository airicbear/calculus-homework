\documentclass[12pt]{article}

\usepackage{amsfonts} % Loads Special Fonts used in Math mode
\usepackage{amssymb}  % Loads Special Symbols used in Math mode
\usepackage{amsthm}   % Allows use of Theorem environment
\usepackage{amsmath}  % More fun stuff used in Math mode\
\usepackage{fancyhdr}

\pagestyle{fancy}
\fancyhf{}
\lfoot{Space Math}
\rfoot{http://spacemath.gsfc.nasa.gov}

\renewcommand{\headrulewidth}{0pt}
\renewcommand{\footrulewidth}{0pt}

\setlength{\oddsidemargin}{-0.1in} \setlength{\textwidth}{6.5in}
\setlength{\topmargin}{-.75in} \setlength{\textheight}{9.75in}

\linespread{1.1}

\begin{document}

\noindent {\sc How to Grow a Planet or a Rain Drop! \hfill 2019-04-04 \hfill Eric Nguyen }

\bigskip

\textbf{Problem 1}: The differential equation for the growth of the mass of a body by accretion is given by Equation 1 and the mass of the body is given by Equation 2

\medskip

Equation 1) $\frac{dM}{dt} = 4 \pi \rho V R(t)^2$ \qquad Equation 2) $M(t) = \frac{4}{3} \pi D R(t)^3$

\medskip

where $R$ is the radius of the body at time $t$, $V$ is the speed of the infalling material, $\rho$ is the density of the infalling material, and $D$ is the density of the body.

\medskip

Solve Equation 2 for $R(t)$, substitute this into Equation 1 and simplify.

\medskip

\textbf{Solution}:  
\begin{align}
R(t) &= \left(\frac{3 M}{4 \pi D}\right)^{1/3} \\
\frac{dM}{dt} &= 4 \pi \rho V \left(\frac{3M}{4 \pi D}\right)^{2/3}
\end{align}

\medskip

\textbf{Problem 2}:  Integrate your answer to Problem 1 to derive the formula for $M(t)$.

\medskip

\textbf{Solution}: 
\begin{align}
\frac{dM}{M^{\frac{2}{3}}} &= 4 \pi \rho V \left(\frac{3}{4 \pi D}\right)^{2/3} dt \\
\int M^{-\frac{2}{3}} dM &= \int 4 \pi \rho V \left(\frac{3}{4 \pi D}\right)^{2/3} dt \\
3M^{\frac{1}{3}} &= 4 \pi \rho V \left(\frac{3}{4 \pi D}\right)^{2/3} t \\
M(t) &= \left(\frac{4}{3} \pi \rho V\right)^3 \left(\frac{3}{4 \pi D}\right)^2 t^3 \\
\end{align}

\clearpage

\textbf{Raindrop Condensation}:  A typical raindrop might form so that its final mass is about 100 milligrams and $D = 1000$ kg/m$^3$, under atmospheric conditions where $\rho = 1$ kg/m$^3$ and $V = 1$ m/sec. How long would it take such a raindrop to condense?

\medskip

\textbf{Solution}:
\begin{align}
100~\text{mg} &= \left(\frac{4}{3} \pi \left(1~\text{kg/m}^3\right) \left(1~\text{m/sec}\right)\right)^3 \left(\frac{3}{4 \pi \left(1000~\text{kg/m}^3\right)}\right)^2 t^3 \\
t^3 &= \frac{0.0001~\text{kg}}{\left(\frac{4}{3} \pi \left(1~\text{kg/m}^3\right) \left(1~\text{m/sec}\right)\right)^3 \left(\frac{3}{4 \pi \left(1000~\text{kg/m}^3\right)}\right)^2} \\
t &\approx 2.88~\text{sec}
\end{align}

\medskip

\textbf{Planet Accretion}:  A typical rocky planet might form so that its final mass is about that of Earth of $5.9 \times 10^{24}$ kg, and $D = 3000$ kg/m$^3$, under conditions where $\rho = 0.0000001$ kg/m$^3$ and $V = 1$ km/sec. How long would it take such a planet to accret using this approximate mathematical model?

\medskip

\textbf{Solution}:
\begin{align}
5.9 \times 10^{24} &= \left(\frac{4}{3} \pi \left(0.000001~\text{kg/m}^3\right) \left(1~\text{km/sec}\right)\right)^3 \left(\frac{3}{4 \pi \left(3000~\text{kg/m}^3\right)}\right)^2 t^3 \\
t^3 &= \frac{0.0001~\text{kg}}{\left(\frac{4}{3} \pi \left(0.000001~\text{kg/m}^3\right) \left(1000~\text{m/sec}\right)\right)^3 \left(\frac{3}{4 \pi \left(3000~\text{kg/m}^3\right)}\right)^2} \\
t &\approx 2.33 \times 10^{13}~\text{sec} \quad \text{or} \quad 738,864.28~\text{years}
\end{align}

\medskip

\end{document}